% Hello! Here s how this works:
%
% You edit the source code here on the left, and the preview on the
% right shows you the result within a few seconds.
%
% Bookmark this page and share the URL with your co-authors. They
% can edit at the same time!
%
% You can upload figures, bibliographies, custom classes and
% styles using the files menu.
%
% This presentation is made with the Beamer package. For tutorials
% and more info, see:
% http://en.wikipedia.org/wiki/Beamer_(LaTeX)
%
% We$re still in beta. Please leave some feedback using the link at
% the top left of this page.
%
% Enjoy!
%
\documentclass{beamer}

%
% Choose how your presentation looks.
%
% For more themes, color themes and font themes, see:
% http://deic.uab.es/~iblanes/beamer_gallery/index_by_theme.html
%
\mode<presentation>
{
  \usetheme{Warsaw}      % or try Darmstadt, Madrid, Warsaw, ...
  \usecolortheme{default} % or try albatross, beaver, crane, ...
  \usefonttheme{professionalfonts}  % or try serif, structurebold, ...
  \setbeamertemplate{navigation symbols}{}
  \setbeamertemplate{caption}[numbered]
} 

\usepackage[english]{babel}
\usepackage[utf8x]{inputenc}
\usepackage{ctex}
% On writeLaTeX, these lines give you sharper preview images.
% You might want to comment them out before you export, though.
\usepackage{pgfpages}
\pgfpagesuselayout{resize to}[%
  physical paper width=8in, physical paper height=6in]

\title{转博面试个人陈述}
\author{张磊}
\institute{西安交通大学数学学院}
\date{2025/5/9}

\begin{document}



% Uncomment these lines for an automatically generated outline.
%\begin{frame}{Outline}
%  \tableofcontents
%\end{frame}

\section{个人简述}
\subsection{基本信息}

\begin{frame}
姓名:张磊

研究方向:非线性泛函分析及其应用

指导教师:尤波

专业课成绩:
\begin{tabular}{|l|l|}

\hline 课程 & 成绩 \\
\hline 非线性分析 & 97\\
\hline  黎曼几何 & 100  \\
\hline 凸分析 & 94 \\
\hline 广义函数与 Sobolev 空间 & 优 \\
\hline 椭圆与抛物方程 & 良 \\
\hline Modern Functional Analysis & 优   \\
\hline 偏微分方程正则性理论 & 良  \\
\hline Carleman 估计以及应用 & 优 \\
\hline Stochastic Differential Equation with Application & 良 \\
\hline
\end{tabular}

\end{frame}

\begin{frame}
    Regularazation technique and bootstrap argument to prove the wellposedness
    of the following 3D MHD Equation

    Riemannian integral can be thought of a net convergence

    \begin{definition}
        A net is a function from a directed set (D, $\leq$) to a topological space
        X.
    \end{definition}
    \begin{equation}
        \int_{a}^{b}f(x)dx
    \end{equation}


\end{frame}


\subsection{硕士经历}

\begin{frame}

学术讲习班:
\begin{tabular}{|c|c|}
\hline 无穷维动力系统天元数学讲习班 & 南京大学\\
\hline \multicolumn{2}{|c|}{Strichartz估计及应用,随机发展方程,正不变吸引子} \\
\hline 分数Brown运动与随机动力系统专题讲习班 & 华南理工大学  \\
\hline \multicolumn{2}{|c|}{分数阶Brown运动,随机吸引子,随机偏微分方程} \\
\hline 第二十届西部高校数学教师暑期学校 & 四川大学\\
\hline \multicolumn{2}{|c|}{高等概率论,随机分析,随机偏微分方程} \\
\hline
\end{tabular}

协办学术会议:
\begin{tabular}{|c|c|} 
\hline 2023年非线性分析与偏微分方程学术会议 & 西安交通大学 \\
\hline \multicolumn{2}{|c|}{变分法,偏微分方程,无穷维动力系统 }\\
\hline
\end{tabular}

\end{frame}


\subsection{科研经历}

\begin{frame}
考虑如下的全空间磁流体MHD方程


试图证明如下定理



\begin{theorem}
    假设 $(v, B) \in L_{\text {loc }, \sigma}^{2}\left(\mathbb{R}^{3} \times[0, T)\right) $ 为MHD方程在  $\left.\mathbb{R}^{3} \times[0, T)\right) $ 
的相对弱解 ,初值  $\left(v_{0}, B_{0}\right) \in L_{\sigma}^{2}\left(\mathbb{R}^{3}\right)$ .若对任意的  $h>0$,$(v, B) $ 满足如下约束
\begin{equation*}
v, B \in L^{p}(h, T ; L^{q}(\mathbb{R}^{3})), \quad \frac{2}{p}+\frac{3}{q}=1, \quad q \in(3, \infty).
\end{equation*}


以及 $ (v, B) \longrightarrow^w\left(v_{0}, B_{0}\right) \in  L^{2}\left(\mathbb{R}^{3}\right)$ .则 $(v, B)$  是一个 Leray-Hopf 弱解. 

\end{theorem}
\end{frame}




 
\end{document}