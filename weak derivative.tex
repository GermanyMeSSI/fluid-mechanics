\documentclass{beamer}

%
% Choose how your presentation looks.
%
% For more themes, color themes and font themes, see:
% http://deic.uab.es/~iblanes/beamer_gallery/index_by_theme.html
%
\mode<presentation>
{
  \usetheme{Warsaw}      % or try Darmstadt, Madrid, Warsaw, ...
  \usecolortheme{default} % or try albatross, beaver, crane, ...
  \usefonttheme{professionalfonts}  % or try serif, structurebold, ...
  \setbeamertemplate{navigation symbols}{}
  \setbeamertemplate{caption}[numbered]
} 

\usepackage[english]{babel}
\usepackage[utf8x]{inputenc}
\usepackage{ctex}
% On writeLaTeX, these lines give you sharper preview images.
% You might want to comment them out before you export, though.
\usepackage{pgfpages}
\pgfpagesuselayout{resize to}[%
  physical paper width=8in, physical paper height=6in]

\usepackage{hyperref}
\hypersetup{
    colorlinks=true,    % 链接着色(true:彩色文本;false:边框)
    linkcolor=blue,     % 内部链接颜色(如公式、章节引用)
    urlcolor=cyan       % 网址颜色
}
\newcommand{\p}{\partial}


\title{对弱导数的几点思考}
\author{力学爱好者}
\institute{XJTU}
\date{2025/9/13}

\begin{document}

\begin{frame}
  \titlepage
\end{frame}


\begin{frame}
   
   
  $L^p=\mathcal{L}^p/\sim$

when we talk about a funtion $u\in L_{loc}^1(\Omega)$ which has a $ith$ weak derivative $g$ also 
in $ L_{loc}^1(\Omega)$, we wrte
\begin{equation*}
    g=\frac{\partial u}{\partial x_i}
\end{equation*}

Note that since $g$ is unique, we can write $\frac{\partial u}{\partial x_i}$ without ambiguity.

Then we have  a operator

\begin{equation*}
    \partial_i :  W^{1,p} \to L^p
\end{equation*}

actually ,we can see that there is a mapping

\begin{equation*}
    \partial_i: \mathbb{W}^{1,p} \to L^p
\end{equation*}

The latter descends to the former operator, and we used the same notation

\end{frame}

\begin{frame}
    
Thus for $[u] \in W^{1, p}$, we define $\partial_i [u] :=\partial_i u $ 
 you should check ion is well defined which means doesnot depend On 
 the representative $u$ !

Then we can dm the linearity of the latter operator !

Hence for $u\in W^{1, p}$, let's say $u=[v]$, hence $u^+=[v^+], u^-=[v^-]$, and 
\begin{equation*}
    \p_iu=\p_i(u^+-u^-)=\p_iv+\p_iv1_{v=0}=\p_iv \quad \in L^p
\end{equation*}

hence
\begin{equation*}
    \p_iv=0 \quad\text{almost everywhere on} \left\{x\in \Omega: v(x)=0\right\}
\end{equation*}

{\small


 you should check that this definition is well defined which means doesnot depend On 
 th





}

\end{frame}


\begin{frame}
  Speaking of complex number $\mathbb{C}$, the precise definition is as follows
  \begin{definition}
    Complex number is a set denoted by $\mathbb{C}$ whose elements are ordered pairs
\begin{equation*}
  \mathbb{C}:=\left\{[a, b]:a, b \in \mathbb{R}   \right\}
\end{equation*}
With the convention that $[a,b]=[c,d]$ iff $a=c, b=d$.
  \end{definition}
Then we define the following addition and mulptication operation to make $\mathbb{C}$ become
a field.
\begin{equation*}
  [a,b]+[c,d]=[a+c,b+d]; [a,b]\times[c,d]=[ad-bc, ac+bd]
\end{equation*}

Then we define $i$ denoting the elemtent $[0, 1]$, it is clear that 
$[0,0]$ and $[1,0]$ are the zero element and unity element in the field.


\end{frame}
\begin{frame}
  There is a natural injection $\mathbb{J}$ between $\mathbb{R} \to \mathbb{C} $
  \begin{equation*}
    J: \mathbb{R}\to \mathbb{C} \quad r\mapsto[r, 0]
  \end{equation*}
Using $J$, we can think of $\mathbb{R}$ as a subset of $\mathbb{C}$, to be precise, when 
we talk about $a\in \mathbb{R}$ as an element in $\mathbb{C}$, we infer the unique element 
$J(a)$!!!!!

Also remember $i=[0,1]$, we have $[a,b]=[a,0]+[0,b]=J(a)+iJ(b)$, we now make
 the convention that $a_\mathbb{C}$=$J(a)$, then 
we write the former expression as $[a,b]=a_\mathbb{C}+b_\mathbb{C}i$



\end{frame}

\begin{frame}
  
Let T be an operator defind on $L^p$, with range in$\mathbf{M}(\mu)/\sim$.
  
Then we define $T\circ u$ where $u\in L^q(0,T;L^p)$ as 
\begin{equation*}
  Tu=T\tilde{u}, \quad [\tilde{u}]=u;
\end{equation*}

Then we can use the axiom of choice to define the following function
\begin{equation*}
  v(t,\bullet)=<Tu(t)>
\end{equation*}

Generating a funtion in $M((0,T)\times\Omega)$  $<<v(t,x)>>$

If $T\tilde{u}$ is $\p_i$, then for $v$, we have the following characteriaztion
{\small

\begin{equation*}
  \int_{\Omega}v(t,x)\phi(x)dx=\int_{\Omega}<\tilde{u}(t)>\p_i\phi(x)dx,\quad \text{for any $t\in (0,T), \phi \in \mathbf{D}(\Omega)$}
\end{equation*}
}

We donot distinguish the case when $\mathbb{M}$ and 
$\mathbf{L}^q$ have different underlying domain.


\end{frame}

\begin{frame}
  Deadkind 分割. 再仔细想一下,  首先我们约定$[a,b]$, 其中$a\in X, b\in Y$ 中的相等, 此时
  再定义笛卡尔乘积为遍历$a\in X$和$b\in Y$

你他妈的如果再这样废话的话,真的有点欠锤了哈,我说实话,  真的得考虑拿个macbook出去了




\end{frame}

\begin{frame}
  
Let $q\in (1,\infty]$, there is a 
countable dense subset $( \{v_j\} \subset C_c(\mathbb{R}^n) )$
in the separable space $L^{q'} (\mathbb{R}^n)$. If $f\in L^r(\mathbb{R}^n)$ where
$1< r < q$
and $g\in L^q(\mathbb{R}^n)$ satisfies the following
\begin{equation*}
  \int_{\mathbb{R}^n}f v_j = \int_{\mathbb{R}^n}g v_j \quad \text{for all $j$}
\end{equation*}

Then whether $f=g$ a.e.?

\end{frame}


\begin{frame}
  
Consider two $\sigma$-finite measure space $(X, A, \mu)$, $(Y, B, \nu)$
For the completion of two space$(X\times Y, (A\otimes B)^{*}, (\mu \otimes \nu)^*)$
. There is a simple observation that for any nonnegative function $f:X\times Y\to [0, +\infty]$
we have if there is a measurable set $X_0\in A$ such that $f_x$ is measurable for all $x\in X_0$
Then the function defined by
\begin{equation*}
  \phi_x=\int_{Y}f(x,y)d\nu
\end{equation*}
is measurable on $X_0$
\begin{proof}
  Note that there exists a funtion $g:X\times Y\to [0, +\infty]$ which is $A\otimes B$ measurable
  such that for almost all $x$, $f(x, \cdot)=g(x, \cdot)$ a.e. on $Y$.
\end{proof}
 Thus there is null set $N$ contained in $X_0$ such 
 that $\phi_x=\int_{Y}g(x,y)d\nu$ on $X_0\setminus N$

\end{frame}

\begin{frame}
  When we have two Banach space $X$ and $Y$, where $X\hookrightarrow Y$, then the notation $L^p(0,T; X) \hookrightarrow L^p(0,T; Y)$
is a little absurbing, as we know in general $X\neq Y$, hence the eqivalence class may arise an issue!

The correct understaning is that we use the following inclusion mapping $i$, and define the mapping 
\begin{align*}
    \tilde{i}: L^p(0,T; X) & \to L^p(0,T; Y) \\
    [u]_{\scriptscriptstyle X} & \mapsto [iu]_{\scriptscriptstyle Y}
\end{align*}



To simplify our notation, we often ignore the subscript and the inclusion mapping $i$, just write $L^p(0,T; X) \hookrightarrow L^p(0,T; Y)$
or $L^p(0,T; X) \subset L^p(0,T; Y)$ even more. But we should know that this is unserious!!
But it is always harmless as we donot change the heart of this idea or this property!

\end{frame}






\end{document}